%\documentclass[10pt]{beamer}		%with pauses
\documentclass[10pt,handout]{beamer} %without pauses

\usepackage[utf8]{inputenc}
\usepackage[english]{babel}
\setbeamersize{text margin left=5mm,text margin right=5mm} 

\usepackage{amssymb,amsmath,amsthm,amsfonts,mathtools}

\mathtoolsset{showonlyrefs}

\let\caron\v %caron accent %https://en.wikibooks.org/wiki/LaTeX/Special_Characters#Escaped_codes

\renewcommand{\a}{\alpha}
\renewcommand{\b}{\beta}
\renewcommand{\c}{\chi}
\renewcommand{\d}{\delta}
\newcommand{\e}{\epsilon}
\newcommand{\f}{\phi}
\newcommand{\g}{\gamma}
\newcommand{\h}{\hbar}
\renewcommand{\i}{\iota}
\renewcommand{\j}{\varepsilon}
\renewcommand{\k}{\kappa}
\renewcommand{\l}{\lambda}
\newcommand{\m}{\mu}
\newcommand{\n}{\nu}
\renewcommand{\o}{\omega}
\newcommand{\p}{\psi}
\newcommand{\q}{\eta}
\renewcommand{\r}{\rho}
\newcommand{\s}{\sigma}
\renewcommand{\t}{\theta}
\renewcommand{\u}{\pi}
\renewcommand{\v}{\varphi}
\newcommand{\w}{\tau}
\newcommand{\x}{\xi}
\newcommand{\y}{\upsilon}
\newcommand{\z}{\zeta}

\newcommand{\A}{\nabla}
\newcommand{\B}[1]{\Big#1}
\newcommand{\C}{\mathbb C}
\newcommand{\D}{\Delta}
\newcommand{\E}{\mathbb E}
\newcommand{\F}{\Phi}
\newcommand{\G}{\Gamma}
\renewcommand{\H}{\mathbb H}
\newcommand{\I}[1]{\tensor[]{}{#1}}
\renewcommand{\L}{\Lambda}
\newcommand{\N}{\mathbb N}
\renewcommand{\O}{\Omega}
\renewcommand{\P}{\Psi}
\newcommand{\R}{\mathbb R}
\renewcommand{\S}{\Sigma}
\newcommand{\T}{\Theta}
\newcommand{\U}{\Pi}
\newcommand{\V}{\wedge}
\newcommand{\X}{\Xi}
\newcommand{\Y}{\mathbb Y}
\newcommand{\Z}{\mathbb Z}

\newcommand{\cA}{\mathcal A}
\newcommand{\cD}{\mathcal D}
\newcommand{\cF}{\mathcal F}
\newcommand{\cG}{\mathcal G}
\newcommand{\cI}{\mathcal I}
\newcommand{\cL}{\mathcal L}
\newcommand{\cM}{\mathcal M}
\newcommand{\cN}{\mathcal N}
\newcommand{\cO}{\mathcal O}
\newcommand{\cS}{\mathcal S}
\newcommand{\cT}{\mathcal T}
\newcommand{\cZ}{\mathcal Z}

\newcommand{\fF}{\mathfrak F}
\newcommand{\fH}{\mathfrak H}
\newcommand{\fg}{\mathfrak g}
\newcommand{\fh}{\mathfrak h}
\newcommand{\fk}{\mathfrak k}
\newcommand{\fm}{\mathfrak m}
\newcommand{\fn}{\mathfrak n}

\newcommand{\dd}{\mathrm{d}}
\newcommand{\ha}[1]{\hat #1}
\newcommand{\la}{\langle}
\newcommand{\ra}{\rangle}
\newcommand{\ti}[1]{\tilde #1}
\newcommand{\bs}[1]{\boldsymbol #1}
\newcommand{\pd}{\partial}
\newcommand{\at}[1]{\vert_{#1}}
\newcommand{\At}[1]{\B\vert_{#1}}
\newcommand{\der}[2]{\frac{\dd#1}{\dd#2}}
\newcommand{\pder}[2]{\frac{\partial#1}{\partial#2}}
\newcommand{\derat}[3]{\der{#1}{#2}\at{#3}}
\newcommand{\pderat}[3]{\pder{#1}{#2}\At{#3}}
\newcommand{\fder}[2]{\frac{\d#1}{\d#2}}
\newcommand{\fderat}[3]{\fder{#1}{#2}\At{#3}}
\newcommand{\id}{\textrm{id}}
\newcommand{\into}{\hookrightarrow}
\newcommand{\onto}{\twoheadrightarrow}
\newcommand{\dmap}{\overset\dd\longrightarrow}
\newcommand{\inv}[1]{#1^{-1}}
\newcommand{\defeq}{:=}


\newcommand{\ham}{hamiltonian}
\newcommand{\lag}{lagrangian}
\newcommand{\eom}{equation of motion}
\newcommand{\eoms}{equations of motion}
\newcommand{\dof}{degree of freedom}
\newcommand{\dofs}{degrees of freedom}
\newcommand{\off}{off-shell}
\newcommand{\on}{on-shell}
\newcommand{\vf}{vector field}
\newcommand{\vfs}{vector fields}
\newcommand{\wrt}{with respect to\ }
\newcommand{\nbh}{neighbourhood}
\newcommand{\ex}{exterior differential}
\newcommand{\ie}{i.e.\@\ }
\newcommand{\eg}{e.g.\@\ }
\newcommand{\st}{s.t.\@\ }
\newcommand{\susy}{supersymmetry}
\newcommand{\sugra}{supergravity}
\newcommand{\rhs}{r.h.s.\@\ }
\newcommand{\lhs}{l.h.s.\@\ }
\newcommand{\rs}{Riemann surface}
\newcommand{\rss}{Riemann surfaces}
\newcommand{\hn}{Hurwitz number}
\newcommand{\hnn}{Hurwitz numbers}
\newcommand{\cpt}{compact}
\newcommand{\cpx}{complex}
\newcommand{\const}{constant}
\newcommand{\holo}{holomorphic}
\newcommand{\branch}{\textsc{Branch}}
\newcommand{\tiff}{if and only if\ }

\newcommand{\YM}{Yang-Mills}
\newcommand{\sYM}{super Yang-Mills}
\newcommand{\SYM}{Super-Yang-Mills}

\renewcommand{\det}{\operatorname{det}}
\newcommand{\End}{\operatorname{End}}
\newcommand{\Lie}{\operatorname{Lie}}
\newcommand{\Map}{\operatorname{Map}}
\newcommand{\Ber}{\operatorname{Ber}}
\newcommand{\diag}{\operatorname{diag}}
\newcommand{\tr}{\operatorname{tr}}
\newcommand{\Vol}{\operatorname{Vol}}
\newcommand{\im}{\operatorname{im}}
\newcommand{\Aut}{\operatorname{Aut}}


\newcommand{\tand}{\quad\text{and}\quad}
\newcommand{\tcomma}{\ ,\qquad}
\newcommand{\tfor}{\quad\text{for}\quad}
\newcommand{\tforall}{\quad\text{for all}\quad}
\newcommand{\imply}{\quad\Rightarrow\quad}
\newcommand{\implied}{\quad\Leftarrow\quad}
\renewcommand{\iff}{\quad\Leftrightarrow\quad}
\newcommand{\twith}{\quad\text{with}\quad}
\newcommand{\twhere}{\quad\text{where}\quad}
\newcommand{\tif}{\quad\text{if}\ }

\newcommand{\tleft}{\text{left}}
\newcommand{\tright}{\text{right}}
\newcommand{\ttop}{\text{top}}
\newcommand{\tint}{\text{int}}
\newcommand{\tst}{\text{st}}
\newcommand{\tss}{\text{ss}}
\newcommand{\tps}{\text{ps}}
\newcommand{\tgm}{\text{gm}}
\newcommand{\tsugra}{\text{sugra}}
\newcommand{\trigid}{\text{rigid}}
\newcommand{\tclosed}{\text{closed}}
\newcommand{\texact}{\text{exact}}
\newcommand{\tCS}{\text{CS}}
\newcommand{\tclass}{\text{class}}

\newcommand{\cclass}{\C_\tclass}

\newcommand{\Rnm}{\R^{n|m}}


\usetheme{Warsaw}
\usecolortheme{default}
\usefonttheme{professionalfonts}

%gets rid of bottom navigation bars
\setbeamertemplate{footline}[frame number]{}

%gets rid of bottom navigation symbols
\setbeamertemplate{navigation symbols}{}

%gets rid of footer
%will override 'frame number' instruction above
%comment out to revert to previous/default definitions
\setbeamertemplate{footline}{}

%environment for equations
\usepackage{xparse}
\NewDocumentCommand \deq { o m }{
  \setlength\abovedisplayskip{4pt}
  \setlength\belowdisplayskip{4pt}
  \setlength\abovedisplayshortskip{0pt}
  \setlength\belowdisplayshortskip{3pt}
\begin{equation}
\begin{aligned}
{}#2
\IfNoValueF{#1}{\label{#1}}
\end{aligned}
\end{equation}
}


\usepackage[backend=biber, style=verbose, sorting=ynt]{biblatex}
\addbibresource{../biblio-beamer.bib}
\setbeamertemplate{bibliography item}{\insertbiblabel}

%------------------------------------------------------------
%This block of code defines the information to appear in the
%Title page
\title[Hurwitz numbers in half infinite wedge space] %optional
{Hurwitz numbers in half infinite wedge space}

\subtitle{From geometry to correlators}

\author[Arthur, Doe] % (optional)
{Andrea Grossutti}

\institute[VFU] % (optional)
{SISSA
  %\inst{1}%
  %Faculty of Physics\\
  %Very Famous University
  %\and
  %\inst{2}%
  %Faculty of Chemistry\\
  %Very Famous University
}

\date[VLC 2021] % (optional)
{%Very Large Conference, 
	May 2022}

%End of title page configuration block
%------------------------------------------------------------



%------------------------------------------------------------
%The next block of commands puts the table of contents at the 
%beginning of each section and highlights the current section:

\AtBeginSubsection[]
{
  \begin{frame}
    \frametitle{Table of Contents}
    \tableofcontents[currentsection,currentsubsection]
  \end{frame}
}
%------------------------------------------------------------

\begin{document}

%The next statement creates the title page.
\frame{\titlepage}


%---------------------------------------------------------
%This block of code is for the table of contents after
%the title page
\begin{frame}
\frametitle{Table of Contents}
\tableofcontents
\end{frame}
%---------------------------------------------------------

\section{From Hurwitz numbers to Burnside's formula}
\subsection{Maps of \rss\ and \hnn}

\begin{frame}

Let $f\colon X\to Y$ be a (non-\const\ \holo) map of (\cpt) and connected \rss\ (hence surjective). \\
For any $y\in Y$, $\inv f(y)=\{x_1,\ldots,x_n\}$, we have $\deg f=\sum_ik_{x_i}$, where $k_x$ denotes the ramification index of $f$ at $x$. We call \emph{ramification profile} of $f$ at $y$ the partition of $d$ given by $(k_{x_1},\ldots,k_{x_n})$. \pause

\begin{theorem}[Riemann-Hurwitz]
	Let $h$ and $g$ denote the genus of $X$ and $Y$ respectively. Then
	\deq{\underbrace{2h-2}_{\c(X)}=(\underbrace{2g-2}_{\c(Y)})\deg f+\sum_{x\in X}\nu_x}
	where $\n_x:=k_x-1$ is the \emph{differential length} of $f$ at $x$.
\end{theorem}
\pause
\begin{definition}
	$f\colon X\to Y$ and $g\colon\ti X\to Y$ are \emph{isomorphic} if there is $\f\colon X\overset\sim\to\ti X$ \st $f=g\circ \f$. 
	
	An \emph{automorphism} of $f\colon X\to Y$ is $\p\colon X\overset\sim\to X$ \st $f=f\circ \p$. \\The group of automorphisms of $f$ is denoted $\Aut(f)$. 

\end{definition}

\end{frame}

%--------------------------------------------------------



\begin{frame}

\begin{definition}
	Let $Y$ RS of genus $g$, $B=\{b_1,\ldots,b_n\}\subset Y$, $d\geq0$, $\q_1,\ldots,\q_n\prt d$. \\We define the \emph{(degree $d$) connected Hurwitz number} to be 
	\deq{H^\circ_{h\overset d\to g}(\q_1,\ldots,\q_n):=\sum_{[f]}\frac1{|\Aut(f)|}}
	and the \emph{(disconnected) (degree $d$) Hurwitz number} to be
	\deq{H^\bullet_{h\overset d\to g}(\q_1,\ldots,\q_n):=\sum_{[f]}\frac1{|\Aut(f)|} \label{eq:disc-H-numb}}
	where both sums runs over isomorphism classes of \emph{Hurwitz covers}, \ie degree $d$ \holo\ maps $f\colon X\to Y$ \st
	\begin{itemize}
		\item $X$ is a \cpt\ RS of genus $h$
		\item the branch locus of $f$ is $B$
		\item the ramification profile of $f$ at $b_i$ is $\q_i$
	\end{itemize}
	In the case of connected Hurwitz numbers we further require $X$ to be connected.
\end{definition}

\end{frame}

%--------------------------------------------------------

\subsection{Maps of \rss\ as ramified covers}

\begin{frame}

Recall that maps of \rss\ can be regarded as ramified covers.

\begin{definition}
	A \emph{ramified cover} is a continuous function between compact topological surfaces $f\colon X\to Y$ with a finite set $B\subset Y$ \st
	\begin{itemize}
		\item$\inv f(B)$ is finite,
		\item $f\colon X\setminus\inv f(B)\to Y\setminus B$ is a covering.
	\end{itemize}
\end{definition}
\pause
Vice-versa, we have

\begin{theorem}[Riemann's Existence Thm.] 
	Let $Y$ \cpt\ \rs, $X^0$ topological surface, $B=\{b_1,\ldots,b_n\}\subset Y$ marked points, $f^0\colon X^0\to Y\setminus B$ covering of finite degree. Then there exists a unique (up to isomorphisms) \cpt\ \rs\ $X$ \st
	\begin{itemize}
		\item $X^0$ is a dense subset of $X$,
		\item $f^0$ extends to a \holo\ map of \rss\ $f\colon X\to Y$.
	\end{itemize}
\end{theorem}

This will be useful to reconstruct $X$ and $f$ given $Y$ and some additional data.

\end{frame}

%---------------------------------------------------------

\begin{frame}

\begin{definition}
	The map $f\colon X\to Y$ is said \emph{$y_0$-labelled} if $y_0\in Y\setminus B_f$ and is chosen an isomorphism $L\colon \inv f(y_0)\overset\sim\to \{1,\ldots,d\}$. We also say that $L$ is a \emph{labelling}. An \emph{isomorphism of $y_0$-labelled maps} $(f,L)$ and $(f',L')$ is an isomorphism of \rss\ $\f\colon X\to X'$ \st 
	\deq{f'\circ\f=f\tand L'\circ\f=L}
\end{definition}
\pause
\begin{definition}
	A $y_0$-labelled map $f\colon X\to Y$ determines a group homomorphism 
	\deq{
		\F\colon\u_1(Y\setminus B_f,y_0)\to S_d\tcomma \g\mapsto\s_\g
	}
	called \emph{monodromy representation}.
\end{definition}



\end{frame}

%---------------------------------------------------------

\begin{frame}

\begin{definition}
	Let $Y$ be a connected \rs\ of genus $g$, $y_0,b_1,\ldots,b_n\in Y$ points, $d\in\Z_{\geq1}$ and $\q_1,\ldots,\q_n\prt d$. A \emph{monodromy representation of type} $(g,d,\q_1,\ldots,\q_n)$ is a group homomorphism $\F\colon\u_1(Y\setminus\{b_1,\ldots,b_n\},y_0)\to S_d$ \st if $\g_k$ is a small loop around $b_k$ then $\F(\g_k)$ has cycle type $\q_k$.
	
	If moreover the subgroup $\im \F\subset S_d$ acts transitively on $\{1,2,\ldots,d\}$ we say that $\F$ is a \emph{connected monodromy representation of type} $(g,d,\q_1,\ldots,\q_n)$.
	
\end{definition}
\pause
We have that $(f,L)\cong(f',L')$ imply $\F=\F'$.

Note also that if two labellings $L,L'$ of $f\colon X\to Y$ are given, then $L'=\s\cdot L$ for some $\s\in S_d$ and $\F'(\g)=\s\cdot\F(\g)\cdot\inv\s$. So monodromy representations of $(f,L)$ and $(f,L')$ are of the same type.

\end{frame}

\begin{frame}

A degree $d$, $y_0$-labelled map $(f,L)$ between \cpt\ \rss\ \st the ramification profile over each branch point is $\q_i$ gives rise to a monodromy representation $\F$ of type $(g,d,\q_1,\ldots,\q_n)$. The monodromy representation will be connected if and only if $X$ is connected. 
\pause

Conversely:

\begin{theorem}
	Let $Y$ be a \rs\ of genus $g$, $\F$ a monodromy representation of type $(g,d,\q_1,\ldots,\q_n)$, $B=\{b_1,\ldots,b_n\}\subset Y$ a finite subset. Then exists a $y_0$-labelled \holo\ map of \rs\ covering $Y$ with branch locus $B$ whose associated monodromy is $\F$. Such map is unique up to isomorphisms of $y_0$-labelled maps. 
\end{theorem}

\end{frame}

\begin{frame}

\emph{\bf Sketch of proof}
	In the proof of this result the Riemann's Existence theorem is fundamental. We construct explicitly the topological space $X_0$ and the covering $f^0\colon X^0\to Y\setminus B$ in such a way that the map $f\colon X\to Y$ given by the Riemann's Existence theorem has the desired monodomy representation. 
	
	Take cycles $\a_1,\ldots\a_g,\b_1,\ldots,\b_g$ in $Y$ generating $\pi_1(Y,y_0)$ all containing a point $p\in Y$ in such a way that $Y\setminus\{\a_1,\ldots\a_g,\b_1,\ldots,\b_g\}$ is the fundamental polygon describing $Y$. Denote by $\g_i$ a loop containing $y_0$, winding once around $b_i$, never around the other elements of $B$, and fully contained in $Y\setminus\{\a_1,\ldots\a_g,\b_1,\ldots,\b_g\}$.
	
	\includegraphics[width=\textwidth]{../figures/CM-fig-7-6.pdf}
	
\end{frame}

\begin{frame}
	\vspace{-10pt}
	\includegraphics[width=0.9\textwidth]{../figures/CM-fig-7-6.pdf}
	
	Consider segments $s_j$ connecting $p$ with the points $b_j$. Open the previous polygon in correspondence of these segments, so that we get a new polygon
	\deq{P:=s_1\ba{s}_1\cdots s_n\ba{s}_n\a_1\b_1\ba{\a}_1\ba{\b}_1\cdots\a_g\b_g\ba{\a}_g\ba{\b}_g}
	Then it suffices to take $d$ copies of $P$ and glue their boundaries appropriately to produce $X^0$ in such a way that the natural projection to $Y\setminus B$ has the desired monodromy representation:\\[-10pt]
	\deq{s_{j,k}&\sim\ba s_{j,\F(\g_j)(k)}\\
		\ba\a_{i,k}&\sim \a_{i,\F(\b_i)(k)}\\
		\b_{i,k}&\sim \ba\b_{i,\F(\a_i)(k)}}\\[-10pt]
	
	\hfill$\blacksquare$
\end{frame}

\begin{frame}

The previous theorem ensures that we have a bijection between isomorphism classes of $y_0$-labeled Hurwitz covers and monodromy representations of type $(g,d,\q_1,\ldots,\q_n)$. \pause Then we can prove

\begin{theorem}
	Let $M^\circ$ (resp $M^\bullet$) be the set of connected monodromy representations (resp. monodromy representations) of type $(g,d,\q_1,\ldots,\q_n)$. Then
	\deq{H^\circ_{h\overset d\to g}(\q_1,\ldots,\q_n)=\frac{|M^\circ|}{d!}}
	and
	\deq{H^\bullet_{h\overset d\to g}(\q_1,\ldots,\q_n)=\frac{|M^\bullet|}{d!}}
	where $h$ is determined by Riemann-Hurwitz. 
\end{theorem}

\emph{\bf Sketch of proof}
	We give the proof in the connected case, the other case is analogous. 

\end{frame}

\begin{frame}


	
	Take $f\colon X\to Y$ Hurwitz cover. Clearly there are $d!$ possible choices of a $y_0$-labelling $L\colon \inv f(y_0)\to \{1,\ldots,d\}$. An automorphism $\f\in\Aut(f)$ is an isomorphism $\f\colon X\overset\sim\to X$ satisfying $f=f\circ\f$. In particular $\f$ gives an isomorphism $(f,L)\cong(f,L')$ where $L'=\f\cdot L:=L\circ\inv\f$. Here $\f\cdot L$ denotes the left action of $\Aut(f)$ on the possible $y_0$-labellings of $f$. Such action is free (\ie $\f\cdot L=L$ imply $\f=\id_X$) so the number of isomorphism classes of $y_0$-labelings of $f$ is $d!/|\Aut(f)|$. 
	
	From the previous theorem isomorphism classes of $y_0$-labelled maps for the given $f$ are in bijection with the distinct monodromy representations arising from $f$ by different labelings of $\inv f(y_0)$. Therefore 
	\deq{m_f=\frac{d!}{|\Aut(f)|}}
	where $m_f$ the number of distinct monodromy representations arising from $f$ by different labelings of $\inv f(y_0)$. So we have
	\deq{H^\circ_{h\overset d\to g}(\q_1,\ldots,\q_n):=\sum_{[f]}\frac1{|\Aut(f)|}=\sum_{[f]}\frac{m_f}{d!}=\frac{|M^\circ|}{d!}}
	where in the last step we used again previous theorem.
	\hfill$\blacksquare$
\end{frame}

\begin{frame}

Although the information carried by connected Hurwitz numbers is usually more interesting for geometrical purposes, it turns out that it is easier to compute the (possibly disconnected) Hurwitz numbers. We will see later how it is possible to recover the connected Hurwitz numbers from the disconnected ones. \pause

\vspace{\baselineskip}

We mention that using some ``degeneration formulas'' (which heuristically correspond to shrink the \rs\ Y producing nodal curves) all disconnected degree $d$ Hurwitz numbers are determined in therms of Hurwitz numbers of the form $H^\bullet_{h\overset d\to0}(\q_1,\q_2,\q_3)$. For this reason (and others) we later restrict our discussion to the case $g=0$. \pause

\vspace{\baselineskip}

Using the last theorem the problem of computing degree $d$ Hurwitz numbers can be translated into a problem in representation theory of the symmetric group $S_d$. In order to show this we need some facts about representation theory. 

\end{frame}

\subsection{Interlude: representation theory of $S_d$}

\begin{frame}

\begin{definition}
	The \emph{group algebra} of the symmetric group $S_d$ is the complex algebra
	\deq{\C[S_d]:=\left\{\sum_{\s\in S_d}a_\s\s\ \big|\  a_\s\in\C\right\}}	
	We define \emph{class algebra} of $S_d$ the center of the group algebra
	\deq{	\cZ\C[S_d]&=\{x\in\C[S_d]\ \big|\ yx=xy\tforall y\in\C[S_d]\}}
\end{definition}

\pause

\begin{definition}
	A \emph{class function} on $S_d$ is a map $\a\colon S_d\to\C$ \st  $\a(\inv hgh)=\a(h)$ $\forall h\in S_d$.\\ Let $\C_\tclass$ denote the vector space of class functions on $S_d$, together with the following Hermitian inner product
	\deq{(\a,\b):=\frac1{d!}\sum_{\s\in S_d}\a(\s)\overline{\b(\s)}=\frac1{d!}\sum_{C\subset S_d}|C|\a(C)\overline{\b(C)} \label{eq:inner-prod-cfunc}}
 	where $\a,\b\in \C_\tclass$ and $\sum_C$ runs over the conjugacy classes of $S_d$. 
\end{definition}

\end{frame}

\begin{frame}

Representations of $S_d$ are described by their characters

\begin{definition}
	Let $\r$ be a representation of $S_d$. The \emph{character} of $\r$ is the class function $\c_\r\in\cclass$ defined by
	\deq{\c_\r(\s):=\tr(\r(\s))}
\end{definition}

\pause

\begin{theorem}
	In terms of the inner product defined in $\cclass$ the characters of the complex irreducible representations of $S_d$ are orthonormal:
	\deq{(\c_{\r_1},\c_{\r_2})=\begin{cases}1\tif \r_1\cong \r_2\\0\tif \r_1\not\cong \r_2\end{cases}}
	where $\r_1,\r_2$ are irreducible representations.  
\end{theorem}

\pause

\begin{theorem}
	To each partition $\l\prt d$ corresponds a unique irreducible representation $V_\l$ of $S_d$. The corresponding character, denoted $\c^\l$, is given by Frobenius formula.
\end{theorem}

\end{frame}

\begin{frame}

For $\q,\l\prt d$, denote $C_\q$ the conjugacy class in $S_d$ of elements of cycle type $\q$, and define the following elements in $\cZ\C[S_d]$:
\deq{c_\q:=\sum_{\s\in C_\q}\s
	\tand
	e_\l:=\sum_{\s\in S_d}\c^\l(\s)\s}
Then one can prove
\deq{\cZ\C[S_d]=\bigoplus_{\q\prt d}\la c_\q\ra_\C=\bigoplus_{\l\prt d}\la e_\l\ra_\C}

\pause

From characters orthogonality (and the fact that $\overline{\c^\l}=\c^\l$) we get
\deq{e_{\l_i}\cdot e_{\l_j}=\begin{cases}e_{\l_i}\tif e_{\l_i}=e_{\l_j}\\0\quad\text{otherwise}\end{cases}}
for this reason $\{e_\l\}$ is called \emph{idempotent basis} of $\cZ\C[S_d]$.

\pause

The formulas for the change of basis are given by the characters
\deq{e_\l=\frac{\dim \l}{d!}\sum_{\q\prt d}\c^\l(C_\q)c_\q
\quad\tand\quad
c_\q=|C_\q|\sum_{\l\prt  d}\frac{\c^\l(C_\q)}{\dim\l}e_\l\label{eq:change-basis-class-alg}}
where $\dim \l:=\dim V_\l$. 

\end{frame}

\subsection{Burnside's formula}

\begin{frame}

Recall that 
\vspace{-3mm}
\deq{H^\bullet_{h\overset d\to g}(\q_1,\ldots,\q_n)=\frac{|M^\bullet|}{d!}}

For $\q=\{\q_1,\ldots,\q_{\ell(\q)}\}\prt d$ where $i\in\Z_{\geq1}$ appears $a_i$ times in the partition, $\sum_i ia_i=d$, the size of the centralizer of $C_\q$ is given by\\[-8pt]
\deq{\fz(\q)=\frac{d!}{|C_\q|}=\prod_{i=1}^{\ell(\q)} a_i! \,i^{a_i}}\\[-8pt]\pause

\begin{definition}
	Let $d\in\Z_{\geq1}$, $\q\prt d$. We define the \emph{kommutator} to be the element
	\deq{\mathfrak K:=\sum_{\q\prt d}\fz(\q)c_\q^2\in\cZ\C[S_d]}
\end{definition}

\begin{theorem}
	\deq{H^\bullet_{h\overset d\to g}(\q_1,\ldots,\q_n)=\frac1{d!}[c_e]\mathfrak K^gc_{\q_n}\cdots c_{\q_2}c_{\q_1}}
	where $[c_e]\mathfrak K^gc_{\q_n}\cdots c_{\q_2}c_{\q_1}$ denotes the coefficient of $c_e$ in $\mathfrak K^gc_{\q_n}\ldots c_{\q_2}c_{\q_1}$.
\end{theorem}
\end{frame}

\begin{frame}

\begin{theorem}
	\deq{H^\bullet_{h\overset d\to g}(\q_1,\ldots,\q_n)=\frac1{d!}[c_e]\mathfrak K^gc_{\q_n}\cdots c_{\q_2}c_{\q_1}}
	where $[c_e]\mathfrak K^gc_{\q_n}\cdots c_{\q_2}c_{\q_1}$ denotes the coefficient of $c_e$ in $\mathfrak K^gc_{\q_n}\ldots c_{\q_2}c_{\q_1}$.
\end{theorem}

By changing basis from the conjugacy basis to the idempotent basis we get

\begin{theorem}[Burnside's Character Formula]
	\deq{H^\bullet_{h\overset d\to g}(\q_1,\ldots,\q_n)=\sum_{\l\prt d}\left(\frac{\dim\l}{d!}\right)^{2-2g}\prod_{i=1}^nf_{C_i}(\l)}
	where $C_i:=C_{\q_i}$ and
	\deq{f_{C_\q}(\l):=|C_\q|\frac{\c^\l(C_\q)}{\dim \l}}
\end{theorem}

\end{frame}

\subsection{The Hurwitz potential}

\begin{frame}

In the following we will restrict ourselves to the case of $g=0$. Recall that there are some degeneration formulas which allows to express all the Hurwitz numbers in terms of those for $g=0$.\pause  

Since Riemann-Hurwitz formula fixes $h$ in terms of $(d,\q_1,\ldots,\q_n)$, we denote
\deq{H^\bullet_d(\q_1,\ldots,\q_n):=H^\bullet_{h\overset d\to 0}(\q_1,\ldots,\q_n)=\sum_{\l\prt d}\left(\frac{\dim\l}{d!}\right)^{2}\prod_{i=1}^nf_{C_i}(\l)}\pause
Let $b$ be the number of branch points which have simple ramification, \ie $\q=(2)$. We denote
\vspace{-10pt}
\deq{H^\bullet_{d,b}(\q_1,\ldots,\q_{m})&:=H^\bullet_d(\q_1,\ldots,\q_{m},\overbrace{(2),\ldots,(2)}^{\text{$b$ times}})\\
&=\sum_{\l\prt d}\left(\frac{\dim\l}{d!}\right)^{2}f_2(\l)^b\prod_{i=1}^{m}f_{C_i}(\l)}
where $f_2:=f_{C_{(2)}}$.\pause Analogous definitions hold for connected Hurwitz numbers, replacing $H^\bullet$ with $H^\circ$. 

\end{frame}

\begin{frame}

Rather than considering the different Hurwitz numbers separately, it worth to collect them together into generating functions.\pause Fix $m\in\Z_{\geq0}$ to be the number of branch points with non-simple ramification profile.

\begin{definition}
	Let $\{p_{i,j}\}$, $q$ and $z$ be some variables, $i\in\{1,\ldots,m\}$, $j\in\Z_{\geq0}$. We define the \emph{Hurwitz potential} to be
	\deq{\fH^\bullet(p_{i,j},q,z):=\sum_{d,b=0}^\infty q^d\frac{z^b}{b!}\sum_{\q_1\prt d}\cdots\sum_{\q_{m}\prt d}p_1^{\q_1}\cdots p_{m}^{\q_{m}}H_{d,b}^\bullet(\q_1,\ldots,\q_m)}
	where for $\q_i=(l_1,\ldots,l_k)\prt d$ we defined
	\deq{p^{\q_i}_i:=(p_{i,1}^{l_1}+\cdots+p_{i,d}^{l_1})\cdots(p_{i,1}^{l_k}+\cdots+p_{i,d}^{l_k})\vspace{-7pt}}
	We also introduce the \emph{modified Hurwitz potential} to be
	\vspace{-2pt}
	\deq{\fh^\bullet_d(\q_q,\ldots,\q_m,z):=\sum_{b=0}^\infty\frac{z^b}{b!}H^\bullet_{d,b}(\q_1,\ldots,\q_m)}
	Analogous definitions hold for the \emph{(modified) connected Hurwitz potential} $\fH^\circ$ and $\fh^\circ$. 
\end{definition}

\end{frame}

\begin{frame}

For fixed $i$, the polynomials of the form $p^\q$ for all partitions $\q\prt d$ form a basis for the space of all homogeneous polynomials of degree $d$ in $d$ variables with rational coefficients. They are called \emph{power sum polynomials}. Therefore, given $\fH^\bullet$, it can be expanded uniquely giving all the Hurwitz numbers. \pause

\vspace{\baselineskip}

The first advantage of considering generating function in place of the single Hurwitz numbers is 

\begin{theorem}
	\vspace{-5pt}
	\deq{\fH^\bullet=e^{\fH^\circ}}
\end{theorem}

\end{frame}

\begin{frame}

In order to simplify the notation, in the following we denote $\c^\l_\q:=\c^\l(C_\q)$. \pause

For any $\l\prt d$, we have the following relation, which can be regarded as corollary of Frobenius formula
\deq{s_\l(p_1,\ldots,p_d)=\frac1{d!}\sum_{\q\prt d}\c^\l_\q|C_\q|p^\q}
where $s_\l$ is the \emph{Schur polynomial} associated to $\l$, it is homogeneous polynomial of degree $d$. \pause Putting formulas together we get
\deq{
	&\fH^\bullet(p_{i,j},q,z)=\\
	&\quad=\sum_{d,b=0}^\infty q^d\frac{z^b}{b!}\sum_{\q_1\prt d}\cdots\sum_{\q_{m}\prt d}p_1^{\q_1}\cdots p_{m}^{\q_{m}}\sum_{\l\prt d}\left(\frac{\dim\l}{d!}\right)^{2}f_2(\l)^b\prod_{i=1}^{m}f_{C_i}(\l)\\
	&\quad=\sum_{d=0}^\infty q^d\sum_{\l\prt d} e^{z f_2(\l)}\left(\frac{\dim \l}{d!}\right)^{2-m}\sum_{\q_1\prt d}\cdots\sum_{\q_{m}\prt d}p_1^{\q_1}\cdots p_{m}^{\q_{m}}\prod_{i=1}^{m}\frac1{d!}|C_{\q_i}|\c^\l_{\q_i}\\
	&\quad=\sum_{d=0}^\infty q^d\sum_{\l\prt d} e^{z f_2(\l)}\left(\frac{\dim \l}{d!}\right)^{2-m}\prod_{i=1}^{m}s_\l(p_{i,1},\ldots,p_{i,d})
}

\end{frame}

\begin{frame}

The case $m=2$ corresponds to the so called \emph{double Hurwitz numbers}, which are the ones we are interested in
\deq{\fH^\bullet(\{p_j,p_j'\},q,z)&=\sum_{d=0}^\infty q^d\sum_{\l\prt d} e^{z f_2(\l)}s_\l(p_1,\ldots,p_d)s_\l(p'_1,\ldots,p'_d)\\
	&=\sum_{\l}q^{|\l|} e^{z f_2(\l)}s_\l(P)s_\l(P')}
We denoted $P:=\{p_1,p_2,\ldots\}$ and $P':=\{p'_1,p'_2,\ldots\}$ (of course $s_\l$ depends only on the first $|\l|$ variables of each set). \pause For $m=2$ we also have
\deq{H^\bullet_{d,b}(\q,\q')&=\frac1{\fz(\q)\fz(\q')}\sum_{\l\prt d}\c^\l_{\q}f_2(\l)^b\c^\l_{\q'}\\
	\fh^\bullet_d(\q,\q',z)&=\frac1{\fz(\q)\fz(\q')}\sum_{\l\prt d}\c^\l_{\q}e^{z f_2(\l)}\c^\l_{\q'}}\pause
Before moving on, notice that for double Hurwitz numbers the Riemann - Hurwitz formula gives
\deq{2h=b+2-\ell(\q)-\ell(\q')}

\end{frame}

\section{Half infinite wedge formalism}

\subsection[Frobenius formula for $C\raisebox{-1pt}{\tiny(2)}$]{Frobenius formula for $C_{(2)}$}

\begin{frame}



Given a partition $\l$, define its rank $r$ to be the length of the diagonal of its Young diagram. Let $a_i$ and $b_i$ be the number of boxes to the right and below of the $i$-th box of the diagonal, reading from the upper right to the lower left.\\
We call $\begin{pmatrix}a_1a_2\ldots a_n\\b_1b_2\ldots b_n\end{pmatrix}$ and $\begin{pmatrix}a'_1a'_2\ldots a'_n\\b'_1b'_2\ldots b'_n\end{pmatrix}$ the \emph{Frobenius notation} and the  \emph{modified Frobenius notation} of the partition respectively, where $a'_i=a_i+1/2$ and $b'_i=b_i+1/2$. 

\vspace{-1.5em}
\[\vspace{-2pt}\begin{matrix}\includegraphics[width=10em]{../figures/FH-pag-51.pdf} &\hspace{-0pt}\raisebox{1.2cm}{$\begin{matrix*}[l]\l=(10,9,9,4,4,4,1)\\\text{Frobenius notation }\begin{pmatrix}9&7&6&0\\2&3&4&6\end{pmatrix}\\\text{modified Frobenius notation }\begin{pmatrix}9.5&7.5&6.5&0.5\\2.5&3.5&4.5&6.5\end{pmatrix}\end{matrix*}$}\end{matrix}\]
\vspace{-1.5em}\pause

\begin{lemma}[Frobenius formula for $C_{(2)}$]
	\vspace{-6pt}
	\deq{\c^\l(C_{(2)})=\frac{\dim\l}{d(d-1)}\sum_{i=1}^r(a_i(a_i+1)-b_i(b_i+1))=\frac{\dim\l}{d(d-1)}\sum_{i=1}^r((a'_i)^2-(b'_i)^2)}
\end{lemma}

\end{frame}

\begin{frame}

From Frobenius formula and $|C_{(2)}|=\begin{pmatrix}d\\2\end{pmatrix}$ it follows that 
\deq{f_2(\l)=\frac{|C_{(2)}|}{\dim \l}\c^\l(C_{(2)})=\frac12\sum_{i=1}^r((a'_i)^2-(b'_i)^2)}\pause

Draw the Young diagram of $\l$ rotated by $135^\circ$ over the real line with opposite orientation as in the picture. 

\[\includegraphics[]{../figures/J-fig-1.pdf}\]

\end{frame}

\begin{frame}

Suppose that $\l$ is made of $k=\ell(\l)$ cycles of lengths $\{\l_1,\ldots,\l_k\}$ where $\l_1\geq\l_2\geq\ldots$. Now consider the ordered set $\{\ti\l_i\}_{i\in\Z_{\geq0}}=\{\l_1,\ldots,\l_k,0,0,\ldots\}$ ending with infinitely many zeros. Then it is clear that black stones are placed in correspondence to elements of 
\deq{\fS_\bullet(\l):=\{\ti\l_i-i+1/2\}\subset\Z+\frac12}
and white stones in correspondence to elements of $\fS_\circ(\l):=(\Z+\frac12)\setminus\fS_\bullet(\l)$. \pause  Moreover, the coefficients in the modified Frobenius notation are given by
\deq{\{a'_i\}&=\fS_\bullet^+(\l):=\fS_\bullet\cap(\Z_{\geq0}+1/2)\\
	\{b'_i\}&=\fS_\circ^-(\l):=\fS_\circ\cap(\Z_{\leq0}-1/2)=(\Z_{\leq0}-1/2)\setminus\fS_\bullet(\l)}\pause
Hence we obtained
\deq{f_2(\l)=\sum_{k\in\fS_\bullet^+}\frac{k^2}2-\sum_{k\in\fS_\circ^-}\frac{k^2}2}\pause
Notice also that
\deq{|\l|=\l_1+\cdots+\l_k=a_1'+\cdots+a_k'+b_1'+\cdots+b_k'=\sum_{k\in\fS_\bullet^+} k-\sum_{k\in\fS_\circ^-}k}

\end{frame}

\subsection{Half infinite wedge formalism}

\begin{frame}

\begin{definition}

Let $V$ be a vector space with basis $\{\und k\}$, $k\in\Z+\frac12$. We define the vector space $\hiw$ to be spanned by vectors
\deq{v_S:=\und{s_1}\V\und{s_2}\V\und{s_3}\V\ldots}
where $S=\{s_1>s_2>\dots\}\subset\Z+\frac12$ is a subset \st both
\deq{S^+=S\setminus\Big(\Z_{\leq0}-\frac12\Big)\tand S^-=\Big(\Z_{\leq0}-\frac12\Big)\setminus S}
are finite. We equip $\hiw$ with the inner product $(-,-)$ in which the basis $\{v_S\}$ (for all possible choices of $S$) is orthonormal. 

\end{definition}\pause

For our purposes $S$ is identified with $\fS_\bullet(\l)$ defined before, we denote by $v_\l$ the vector in $\hiw$ corresponding to $S=\fS_\bullet(\l)$. In this case $S^+=\fS_\bullet^+$ and $S^-=\fS_\circ^-$, and finiteness condition simply corresponds to the fact that we have finitely many black stones on the left of the origin and finitely many white stones on the right or the origin, which is automatic for $\fS_\bullet(\l)$. 

\end{frame}

\begin{frame}

\begin{definition}

For all $k\in\Z+\frac12$ define the operators $\p_k$ and $\p_k^*$ on $\hiw$ by 
\deq{\p_k(v):=\und k\V v\tand (v',\p^*_kv)=(\p_kv',v)}
where $v,v'\in\hiw$.
\end{definition}\pause

From the definitions we have
\deq{\p_j\p_k^*+\p_k^*\p_j=\d_{jk}
	\tand
	\p_j\p_k+\p_k\p_j=\p_j^*\p_k^*+\p_k^*\p_j^*=0}
which are known as \emph{fermionic commutation relations}. \pause

These operators are related to the usual creation and annihilation operators for the \emph{Fermi sea}, by identifying black stones with electrons and white stones with empty energy levels (but in physical literature $\p_k$ and $\p_k^*$ are exchanged). 

Finiteness condition amounts to considering state which are finite energy excitation of the vacuum state. \pause

We define \emph{vacuum} the vector $\vac:=\und{-\frac12}\V\und{-\frac32}\V\und{-\frac52}\V\ldots$, where all stones are black (resp. white) on the right (resp. left) of the origin. This is exactly the vacuum in the Fermi sea model.

\end{frame}

\begin{frame}

\begin{definition}

Introduce the \emph{normal ordered product}
\deq{\no{\p_j\p_k^*}\ \ :=\begin{cases}\p_j\p_k^*\quad&k>0\\-\p_k^*\p_j\quad&k<0\end{cases}}

\end{definition}\pause

For $k>0$ (resp. $k<0$), $\no{\p_k\p_k^*}$ gives 1 (resp. -1) if we have a black (resp. white) stone in $k$ and zero otherwise.\\
For $j\neq k$ the normal ordered product is the ordinary product due to the fermionic commutation relation $\no{\p_j\p_k^*}\ =\p_j\p_k^*=-\p_k^*\p_j$.\\\pause
The effect of $\no{\p_j\p_k^*}\ $ for $j\neq k$ is to take the black stone in $k$ and move it to the position $j$, unless there is no black stone in $k$ or the position $j$ is already occupied: in such cases it gives the zero vector. \\\pause
Finiteness condition ensures that any operator of the form $\sum_{j,k}a_{jk}\no{\p_j\p_k^*}$ is well defined for any choice of the coefficients $a_{j,k}$.

\end{frame}

\begin{frame}

\begin{definition}
\deq{\cF_2=\sum_{k\in\Z+\frac12}\frac{k^2}2\no{\p_k\p_k^*}}
\end{definition}

It satisfies\\[-15pt]
\deq{\cF_2v_\l=f_2(\l)v_\l}\vspace*{-\baselineskip}\pause

\begin{definition}
We define the following operators, called \emph{energy operator} and \emph{charge operator} respectively\\[-15pt]
\deq{H:=\sum_{k\in\Z+1/2}k\no{\p_k\p_k^*}
	\qquad
	C:=\sum_{k\in\Z+1/2}\no{\p_k\p_k^*}}
\end{definition}

The physical interpretation of these operators is obvious when we regard a black stone at position $k>0$ as an electron of energy $k$ and charge $+1$ and a white stone at position $k<0$ as a positron of energy $-k$ and charge $-1$. \pause

We have\\[-15pt]
\deq{Hv_\l=|\l|v_\l}

\end{frame}

\begin{frame}

For the charge operator we have 
\deq{Cv_S=(|S^+|-|S^-|)v_s}\pause
Suppose $Cv_S=c\,v_S$, then $|S^+|=|S^-|+c$. Pick $k\in S$, $k<\min (S^-)$, this means that the stone at $k$ and all those on its right are black. Then $|S^+|=|S^-|+c$ implies that $k=s_{-(k-1/2)+c}$. Hence
\deq{c=\lim_{i\to+\infty}\Big(s_i+i-\frac12\Big)}\pause
In particular, all vectors corresponding to Young diagrams ($S=\fS_\bullet(\l)$) have zero charge. Conversely, any vector of zero charge can be obtained from a Young diagram by taking the partition $\l_i=s_i+i-\frac12$ (omitting the infinitely many $\l_i$ which vanish). We denote by $\hiwz$ the subspace of $\hiw$ of zero charge. \pause From this we get that\\[-15pt]
\deq{\{v_\l\,\big|\,\l\prt d\}}
is a basis for the subspace of $\hiwz$ of energy $d$. Considering all positive values of $d$ we get a basis of the whole charge zero subspace $\hiwz$. 

\end{frame}

\begin{frame}

\begin{definition}
For $n\in\Z_{\neq0}$ define\\[-18pt]
\deq{\a_n:=\sum_{k\in\Z+\frac12}\no{\p_{k-n}\p^*_k}}
\end{definition}\pause

They satisfy $\a_n\und k=\und{k-n}$ and the \emph{Heisenberg commutation relations}
\deq{[\a_n,\a_m]=n\d_{n+m,0}}
From the definition it follows that $\a_n$ and $\a_{-n}$ are adjoint for every $n\in\Z_{\neq0}$.\pause

The effect of this operator in a certain configuration of stones (or electrons) is the following. Pick one and move it to the right by $n$ positions. If the new position is a white stone, make it black and replace the initial black stone by a white one. If the new position is occupied by a black stone, then the operator gives the zero vector (Pauli principle). Then repeat this for all black stones and sum all the resulting vectors to give the final state.

\end{frame}

\begin{frame}

\begin{definition}
For any partition $\q=\{\q_1,\ldots,\q_{\ell(\q)}\}\prt|\q|$ define
\deq{\a_{\q}:=\prod_{i=1}^{\ell(\q)}\a_{\q_i}
	\tand
	\a_{-\q}:=\prod_{i=1}^{\ell(\q)}\a_{-\q_i}} 
\end{definition}\pause

The Heisenberg commutation relation ensures that the ordering of the multiplied operators is not relevant and also that $\a_\q$ and $\a_{-\q}$ are adjoint. \pause Note also that $\a_{\q}$ (resp. $\a_{-\q}$) decreases (resp. increases) the energy of states by $|\q|$. \pause Moreover, it is possible to prove that 
\deq{\{\a_{-\q}\vac\,\big|\,\q\prt d\}}
is a basis for the subspace of $\hiwz$ of energy $d$, orthogonal \wrt the inner product $(-,-)$.

\end{frame}

\begin{frame}

The two bases of $\hiwz$ can be related as follows. Let $\l$ be any partition and $n\in\Z_{>0}$. Then graphically we can see that 
\deq{\a_{n}v_\l=\sum_{\l'<\l\atop{\text{$\l/\l'$ skew hook}\atop{|\l/\l'|=n}}}(-1)^{r(\l/\l')-1}v_{\l'}}
where $r(\l/\l')$ is the number of rows of $\l$ touched by $\l/\l'$. Indeed moving a black stone corresponds to remove a skew hook of length $n$ from the partition (the strip of length 2 in the picture).

\[\includegraphics[width=\textwidth]{../figures/J-fig-2.pdf}\]

\end{frame}

\begin{frame}

The two bases of $\hiwz$ can be related as follows. Let $\l$ be any partition and $n\in\Z_{>0}$. Then graphically we can see that 
\deq{\a_{n}v_\l=\sum_{\l'<\l\atop{\text{$\l/\l'$ skew hook}\atop{|\l/\l'|=n}}}(-1)^{r(\l/\l')-1}v_{\l'}}
where $r(\l/\l')$ is the number of rows of $\l$ touched by $\l/\l'$. Compare to
\begin{lemma}[\emph{Murnaghan-Nakayama rule}]
	\deq{\c^\l_{\{(n),\q\}}=\sum_{\l'<\l\atop{\text{$\l/\l'$ skew hook}\atop{|\l/\l'|=n}}}(-1)^{r(\l/\l')-1}\c^{\l'}_\q}
\end{lemma}\pause
Using these identities we get that for any two partitions $\l$ and $\q$ \st $|\l|=|\q|$ 
\deq{\a_{\q} v_\l=\c_\q^\l v_\emptyset}
and taking the adjoint
\deq{\a_{-\q}\vac=\sum_{\l\prt |\q|}\c_\q^\l v_\l}

\end{frame}


\section{Hurwitz numbers in the half infinite wedge space}

\subsection{Hurwitz numbers using correlators}

\begin{frame}

For $\q=\{\q_1,\ldots,\q_{\ell(\q)}\}\prt d$ and $\q'=\{\q'_1,\ldots,\q'_{\ell(\q')}\}\prt d$ we obtain
\deq{\big(\vac,\a_\q \cF_2^b\a_{-\q'}\vac\big)
	&=\sum_{\l\prt d}\c_{\q'}^\l\big(\vac,\a_\q \cF_2^b v_\l\big)
	=\sum_{\l\prt d}f_2(\l)^b\c_{\q'}^\l\big(\vac,\a_\q v_\l\big)\\
	&=\sum_{\l\prt d}\c_\q^\l f_2(\l)^b\c_{\q'}^\l\big(\vac,\vac\big)
	=\sum_{\l\prt d}\c_{\q}^{\l}f_2(\l)^b\c_{\q'}^{\l}}\pause
Therefore
\deq{H^\bullet_{d,b}(\q,\q')&=\frac1{\fz(\q)\fz(\q')}\sum_{\l\prt d}\c^\l_{\q}f_2(\l)^b\c^\l_{\q'}=\frac1{\fz(\q)\fz(\q')}\big\la\a_\q \cF_2^b\a_{-\q'}\big\ra}
where the \emph{correlator} of the operator $A$ is defined by $\la A\ra:=(v_\emptyset,Av_\emptyset)$. \pause Similarly
\deq{\fh^\bullet_d(\q,\q',z)&=\frac1{\fz(\q)\fz(\q')}\sum_{\l\prt d}\c^\l_{\q}e^{z f_2(\l)}\c^\l_{\q'}=\frac1{\fz(\q)\fz(\q')}\big\la\a_\q e^{z\cF_2}\a_{-\q'}\big\ra}

\end{frame}

\begin{frame}

To rewrite the Hurwitz potential we should be able to recover the Schur polynomials from the half infinite wedge formalism. \pause

\begin{definition}
Given a sequence $t=(t_1,t_2,\ldots)$ define
\deq{\G_\pm(t)=\exp\Bigg(\sum_{n=1}^\infty t_n\a_{\pm n}\Bigg)}
\end{definition}

Note that $\G_+(t)$ and $\G_-(t)$ are adjoint. \pause
\begin{lemma}
	Consider a set of variables $P=(p_1,p_2,\ldots)$ and denote $p^{(k)}=p_1^k+p_2^k+\cdots$. Then\\[-15pt]
	\deq{\G_-\bigg(p^{(1)},\frac{p^{(2)}}2,\frac{p^{(3)}}3,\cdots\bigg)\vac=\sum_\l s_\l(P)v_\l}
	where $\sum_\l$ runs over all possible partitions (of any number).
\end{lemma}

\end{frame}

\begin{frame}
Introduce the following abbreviations
\deq{\G_+:=\G_+\bigg(p^{(1)},\frac{p^{(2)}}2,\frac{p^{(3)}}3,\cdots\bigg)
	\tand
	\G_-:=\G_-\bigg(p'^{(1)},\frac{p'^{(2)}}2,\frac{p'^{(3)}}3,\cdots\bigg)}
where $P=(p_1,p_2,\ldots)$, $P'=(p'_1,p'_2,\ldots)$. \pause
Using the lemma we have
\deq{&\big(\vac,\G_+q^He^{z \cF_2}\G_-\vac\big)
	=\sum_{\l,\l'}s_\l(P)s_{\l'}(P')\big(v_\l,q^He^{z \cF_2}v_{\l'}\big)\\
	&\qquad=\sum_{\l,\l'}q^{|\l'|}s_\l(P)e^{z f_2(\l')}s_{\l'}(P')\big(v_\l,v_{\l'}\big)
	=\sum_{\l}q^{|\l|}s_\l(P)e^{z f_2(\l)}s_{\l}(P')}\pause
Hence
\deq{\fH^\bullet(\{p_j,p_j'\},q,z)=\big\la\G_+q^He^{z \cF_2}\G_-\big\ra}

\end{frame}

\begin{frame}

Summarizing, we obtained the following formulas
\deq{H^\bullet_{d,b}(\q,\q')&=\frac1{\fz(\q)\fz(\q')}\big\la\a_\q \cF_2^b\a_{-\q'}\big\ra\\
	\fh^\bullet_d(\q,\q',z)&=\frac1{\fz(\q)\fz(\q')}\big\la\a_\q e^{z\cF_2}\a_{-\q'}\big\ra\\
	\fH^\bullet(\{p_j,p_j'\},q,z)&=\big\la\G_+q^He^{z \cF_2}\G_-\big\ra}\pause
From this expression it was shown that the Hurwitz potential $\fH^\bullet$ is the $\w$-function for the Toda lattice hierarchy of Ueno and Takasaki, implying (infinitely) many recursive relations on $\fH^\bullet$. More about this in a forthcoming seminar!
\end{frame}

\subsection{Explicit computation of Hurwitz numbers using commutation relations}

\begin{frame}

\begin{definition}
	For $r\in\Z$ define\\[-15pt]
	\deq{\cE_r(z):=\sum_{k\in\Z+\frac12}e^{z(k-r/2)}E_{k-r,k}+\frac{\d_{r,0}}{\varsigma(z)}}
	where $E_{ij}=\ \no{\p_j\p_k^*}$ and $\varsigma(z):=e^{z/2}-e^{-z/2}$.
\end{definition}\pause

The exponent $r/2$ in the definition is used in order to have
\deq{\cE_r(z)^*=\cE_{-r}(z)}
The operators $\cE$ satisfy the following commutation relation
\deq{[\cE_a(z),\cE_b(w)]=\varsigma(\det\left[\begin{smallmatrix}a&z\\b&w\end{smallmatrix}\right])\cE_{a+b}(z+w)}
The operators $\cE$ specialize to the standard bosonic operators on $\hiw$
\deq{\a_k=\cE_k(0)\tcomma k\neq0}\vspace*{-\baselineskip}\pause
\begin{lemma}
	For $n\in\Z_{\geq1}$
	\deq{e^{z\cF_2}\a_{-n}e^{-z\cF_2}=\cE_{-n}(nz)}
\end{lemma}

\end{frame}

\begin{frame}

Using $\cF_2\vac=0$ we have
\deq{\fh^\bullet_d(\q,\q')&=\frac1{\fz(\q)\fz(\q')}\big\la\a_\q e^{z\cF_2}\a_{-\q'}\big\ra
	=\frac1{\fz(\q)\fz(\q')}\big\la\prod_{i=1}^{\ell(\q)}\a_{\q_i}\prod_{j=1}^{\ell(\q')}\big(e^{z\cF_2}\a_{-\q'_j}e^{-z\cF_2}\big)\ra\\
	&=\frac1{\fz(\q)\fz(\q')}\big\la\prod_{i=1}^{\ell(\q)}\cE_{\q_i}(0)\prod_{j=1}^{\ell(\q')}\cE_{-\q'_j}(z\q'_j)\big\ra}\pause
Using the commutation relation for the operators $\cE$ it is possible to compute the correlator in the previous formula by moving the operators with negative energy on the right and those of positive energy on the left. Then only operators of zero energy survive, so we get a sum of terms of the form $\vars(m_1z)\vars(m_2z)\cdots\cE_0(n_1z)\cE_0(n_2z)\cdots$ for some positive integers $m_1,m_2,\ldots,n_1,n_2,\ldots$. \pause
Now recall
\deq{
	\cE_0(nz)=\sum_{k\in\Z+\frac12}e^{nzk}E_{k-r,k}+\frac1{\vars(nz)}}  
and $E_{k-r,k}\vac=0$ for all $k,r$. Hence $\la\cE_0(nz)\ra=\inv\vars(nz)$ and we get as final result a sum of terms of the form $\vars(m_1z)\vars(m_2z)\cdots\inv\vars(n_1z)\inv\vars(n_2z)\cdots$.  

\end{frame}

%%%%%%% ------ SLIDE ------ %%%%%%%
\begin{frame}[allowframebreaks]
\frametitle{Bibliography}
	
	\nocite{*}
	\printbibliography


\end{frame}


\end{document}